\documentclass{article}
\usepackage[none]{hyphenat}
\usepackage{geometry,multicol,setspace,flushend}
\geometry{letterpaper, left=10mm, right=10mm, top=10mm, bottom=10mm}

\setlength{\columnsep}{1cm}
\setlength{\parindent}{0pt}
\pagenumbering{gobble}

\begin{document}
\begin{multicols}{2}
\begin{spacing}{1.1}
% \enlargethispage{\baselineskip}

%%%%%%%%%%%%%%%%%%%%%%%%%%%%%%%%%%%%
% 25 mins presentation + 5 mins Q&A
%%%%%%%%%%%%%%%%%%%%%%%%%%%%%%%%%%%%

\subsection*{\small Greeting 
  \noindent\rule{0.1\textwidth}{1pt}\small(1)}
Hello, everyone.

Thank you for coming to my presentation.

This presentation is for my thesis proposal defence.

Today I am talking about the dark side of the food business. 
It is food loss and waste in a restaurant.

Especially this study is a case study and
includes its definition and measurement, potential factors, statistical model, and its impacts on society and the environment.


\vspace{1em}
\subsection*{\small Reason  \noindent\rule{0.1\textwidth}{1pt}\small(2)}

Why I chose this topic is that I have worked at restaurants for over 5 years, almost 10 years;
as a server, kitchen staff, and management side.

- As a server, 
I was watching much still edible food going back and dumped.

- As a chef, 
I was trying new menu ideas from food that was being thrown away.

- As a management side, 
I have been managing inventory and training employees to eliminate food waste.

And now I have the opportunity to step out of the restaurant 
and do some research on food loss and waste.

% \vspace{1em}
% \underline{Statement of Purpose}
% 
% The purpose of this study is 
% to share readers with a better understanding of the food loss and waste as one of a case study.
% 
% Especially, 
% how to measure the food loss and waste in a restaurant,
% what is the potential factor, and 
% what is the consequence of it.
% 
% \vspace{1em}
% \underline{Audience}
% 
% So, I suppose that this presentation provides information about
% one case of a restaurant in Prince George's around food loss and waste.
% 
% Therefore, 
% I expect my audience will be 
% those who are researching food loss and waste, 
% managers running in a restaurant, 
% and public policymakers who are working on food loss.

\vspace{1em}
\subsection*{\small Today's Menu \noindent\rule{0.1\textwidth}{1pt}\small(3)}

This is today's special menu.

- First introduction. 

It will show you some facts about food loss and waste, 
and then my research questions will be there.

- Second, Literature review. 
This will cover previous research on food loss and waste.

- Third, Methods. 
This section explains how I actually measure food loss and waste.

- Finally, I will show you what can be inferred from the results of the measurements.

\vfill\null
\columnbreak
\subsection*{\small Introduction \noindent\rule{0.1\textwidth}{1pt}\small(4)}

I would like to start with some numbers about food loss and waste.

Actually food loss and waste is a global and local issues.

What I mean, it happens everywhere in the food supply chain.

According to FAO, Food and Agriculture Organization,
it is estimated that one-third or one-fourth of food is lost or wasted.

And about 1.3 billion tons of food loss and waste are generated each year, 
and it is going to grow every year. (44\% per year).

In Canada,
it creates about 35 million tons of food waste, and
it said that Canada was the largest food waste generator per person in western countries in 2016.

In British Columbia, 
almost 40\% of waste sent to landfills is organic waste.
And then the majority of it is coming from domestic or households.

\subsection*{\small Little Restaturant Studies \noindent\rule{0.1\textwidth}{1pt}\small(5)}

Therefore, most studies of food loss and waste are basically focusing on households.

So, there is little research on the food supply side.

Especially, 
even though the production or distribution sectors are still advancing,
the food service sector is not.

And I found an interesting number from a provincial report that is
food waste in the food service sector is between 1\% and 15.5\%,

and under some sets of conditions, it will be up to 52\%.

\subsection*{\small Research Questions \noindent\rule{0.1\textwidth}{1pt}\small(6)}

Since I have worked in several restaurants,

I would like to test this number, and

I came up with the following questions.

- What is the average volume of food that is wasted during processing and consumption in restaurants?

- What is the extent of food wastage in Japanese restaurants in Prince George?

- What are the main factors contributing to food loss and waste?

- To what extent is a social or environmental impact from food loss and waste
generated by a single restaurant?

- What approaches are Japanese restaurant operators taking to reduce food waste generation?

\vfill\null
\columnbreak
\subsection*{\small Definitions \noindent\rule{0.1\textwidth}{1pt}\small(7)}

Now let's look at the past studies about food loss and waste.

So, what exactly is the food loss and waste?

Actually, there is no universally accepted definition.

In other words, the definition is different depending on the researchers or research organizations.

Here are some definitions used by four typical organizations: FAO, EU, North America governments.

\vspace{1em}
- First, FAO is one of the international organizations in the united nation.
FAO defines food loss as

"Decrease in weight (dry matter) or quality (nutritional value) of food that was originally produced for human consumption."

and food waste is 

"The disposal of food suitable for human consumption after it has been stored in a decomposed state or beyond its expiration date."

\vspace{1em}

- Second, the Food waste definition by European Union is 

"Any food, whether edible or inedible parts, removed from the Food Supply Chain to be disposed or recovered"

and the unique point is that all food loss and waste refer to food waste, 
so there is no food loss definition.

\vspace{1em}

Third, for the North American governments, US and Canada, 
they have a similar definition.

It is actually opposite to the EU's definition,
food loss is a decrease in food products throughout the food supply chain.

And food waste is still consumable food but not consumed, which is overlapped with food loss.

\vspace{1em}

It is difficult to explain in words only, 
so the next slide describes each difference in a diagram.

% \subsection*{\small Definitions Diagram 
\noindent\rule{0.1\textwidth}{1pt}\small(8)
% }

Food loss and waste are basically grouped into food loss and food waste, 
in the food supply chain: 
which is from food production, handling and storage, processing, distribution, and to consumption.

\vspace{1em}

According to the FAO's definition, 
food loss means the removed food from food production to processing, 
while food waste is generated from distribution to consumption level.

\vspace{1em}

In the EU, 
there is no definition of food loss; so
all food removed from the food supply chain is Food Waste.

\vspace{1em}

In contrast, 
the US and Canada use food loss as unconsumed food in the food supply chain;
while food waste is similar to FAO's range.

\vspace{1em}

The academic studies tend to define 1-4 as food loss and the last 5, 
i.e., the consumer level, as food waste.

\vspace{1em}

As shown above, 
the criteria for Food Loss and Food Waste are different 
between research organizations and countries, 
but in general, 
Food Loss refers to food removed from the upstream processes of the food supply chain,
while Food Waste is thrown away in the downstream processes.

% \vfill\null
% \columnbreak
% \subsection*{\small Soup? 
\noindent\rule{0.1\textwidth}{1pt}\small(9)
% }

Except EU, 
all these-above definitions of Food loss and waste point to solid waste:
that is food scraps, peels or leftovers.

How about soup?
Is soup not food?

Well, I think that liquid form food waste, such as oil or soup, cannot be ignored.

\vspace{1em}

This is because, for one thing, 
discharging the soup may put stress on sewage treatment or system and 
also on the environment, such as damaging soil, rivers, and the air.

Second thing is that it is equivalent to wasting the food used to prepare the soup, or the liquid form of food waste.

Many studies have ignored it because it is difficult to measure, 
but this study tried to measure its weight as a trial study.

\noindent\rule{0.2\textwidth}{1pt}(10)

To summarize the previous definitions and use them to our case study, 
the following diagram is our food loss and waste definition.

\vspace{1em}

Food loss is the food decrease in weight generated by the food provider, 
such as cooking scraps or poor storage.

And food waste is the food weight loss created by consumers in the restaurant, 
such as a bowl or plate leftover.

% \subsection*{\small Measurements 
\noindent\rule{0.1\textwidth}{1pt}\small(11)
% }

Well, how can we capture the food loss and waste in a restaurant?

In past studies, 
there are five main methods of measuring food loss and waste.

\vspace{1em}

% 1
- The first method is self-report.

This method involves recruiting volunteers, 
asking them questions about 
how much food waste they leave per meal or per a day, and ask to record it.

This method has relatively low-cost, but high rate of drop out is a problem.

\vspace{1em}

% 2
- Next method is a questionnaire or paper interviews.

This method collects food loss and waste data by interviewing each person or research site diners.

This method of survey is a cost-effective way,

But, people usually tend to underestimate their food loss and waste, so the estimation is often biased.

\vspace{1em}

% 3
- Waste composition method is one of the good methods. 

It is used to understand the different components that makeup food loss and waste.

This method has a huge advantage in understanding the reason why it might happen;
however, this needs a lot of costs: the requirement of labs and special skills.

\vspace{1em}

% 4
- Mass balance method is also called material flow analysis or input-output analysis.

This method estimates food loss and waste 
by measuring inputs, such as raw food materials, 
and outputs like, food products along with changes in food weight. 

% When making estimates,
% This method makes various assumptions about food processing and creates an input-output model.

This method can be used to estimate food loss and waste 
where it is not available for reliable measurements.

\vspace{1em}

% 5
The final method is a direct measurement.

This method is the most popular method and, 
as you can guess, 
it measures the weight or volume of food loss and waste directly.
This method might have a relatively high cost, 
but most accurate and does not need high skills.

\vspace{1em}

In this research, 
we are interested in the volume of food loss and waste, 
so I will use number 5, which is direct measurement.

% \vfill\null
% \columnbreak
\subsection*{\small Causes \noindent\rule{0.1\textwidth}{1pt}\small(12)}

Now we can think of the cause side of food loss and waste.

What makes food loss and waste in a restaurant?

Actually, the causes or the factors are complex and complicated.
However, I will introduce a Finnish study that classified eight factors, which are best described.

\vspace{1em}

% 1. 
1. The first factor is society.

Society as a whole form the framework for people and businesses to operate.

In other words, 
the culture, norms, rules, regulations or laws influence people's behaviour, and it might affect food waste.

\vspace{1em}

% 2
2. A company's business concept or model is an idea of how it operates in the market.

For example, comparing the à la carte style and buffet style,

Buffet-style restaurants need to prepare more food, 
and if it is not consumed by the customers, they are thrown away.

\vspace{1em}

% 3.
3. 
The food purchasing process is another important component of food loss.

For example, when using frozen items, they might be of poor quality compared to fresh items, which might be an increase in plate leftovers. 
But on the other hand, if using fresh items only, 
they are easily spoiled and thrown away in the kitchen.

\vspace{1em}

% 4.
4. 
A restaurant’s management influences the amounts of kitchen waste, serving loss and plate leftovers.

This is my experience, for example,
I have created new menus by combining foods usually thrown away with other foods.
This type of menu creation or devising reduces food loss.

\vspace{1em}

% 5.
5. 
Having new kitchen staff may generate food loss. 

This is because they tend to make mistakes in cooking, which may lead to throwing food away. 
Mistakes may also be due to reading a recipe incorrectly or carelessly. 

\vspace{1em}

% 6.
6. 
Dine-in customers create food waste in restaurants. 

If the quality or taste of the food is poor, 
or customers find that they are not hungry, 
even or if the food does not meet their expectations, 

they are easily left on the plate.

\vspace{1em}

% 7.
7. 
The presence of competitors also has the potential impact on food waste in restaurants.

For example, 
if a new rival restaurant opens or a rival restaurant changes its menu, 
it becomes difficult to predict the amount of food needed by its own restaurant, 
and prepared food might end up in the trash.

\vspace{1em}

% 8.
8. communication, or lack of it, might have an impact on the amount of trash in the kitchen and on the dishes.

Communication is essential not only among kitchen staff but also with customers and within the supply chain.

Without communication, 
it is difficult to share information, meet customer expectations, and respond to defective or misdelivered products, which can reduce the amount of food loss and waste.

% \vspace{1em}

% \vfill\null
% \columnbreak
% \subsection*{\small Other factors? 
\noindent\rule{0.1\textwidth}{1pt}\small(13)
% }

These categorized factors are well-defined, 
and in my experience, I have the same experience.

But, unfortunately, these factors do not change substantially during this study.

For example, 
no new restaurants have opened in the neighbourhood, 
and no change in business management, procurement and employees have not changed.

\vspace{1em}

However, in addition to these factors, 
I would say that calendar and weather conditions may influence food waste.

For example, 
this is not only in the food business sector, 
but also in the retail business, 
usually, total sales change from month to month and week to week, or day to day. 

This is called the calendar effect.

\vspace{1em}

In addition, 
sales may also change depending on the weather.

As a simple example, 
ice cream usually sells well in hot weather but not in cooler weather.

People's purchasing patterns change because of weather conditions. 

These changes in purchasing patterns may also affect food waste.

\vspace{1em}

Therefore, in this study, we would like to check the calendar and weather effects on food loss and waste.

\subsection*{\small Model \noindent\rule{0.1\textwidth}{1pt}\small(14)}

Now, 
to test how much the effects of the calendar and weather affect the amount of food loss and waste, 
this study will use a regression analysis.

\vspace{1em}
It considers food loss and waste as the dependent variable 
and the explanatory variables are the weather conditions and the day of the week, or sales for the day.

\vspace{1em}

Both variables are measured in continuous or discrete numbers,
and I would like to check the connection between the dependent and the independent variables.

so, we believe a regression analysis is valid.


% \noindent\rule{0.2\textwidth}{1pt}(14)
% 
% A typical linear regression analysis sets the following six assumptions.
% 
% However, there might be a problem here.
% 
% It is possible that using typical regression analysis may generate spurious regressions.
% 
% The main reason is that the food loss and waste data is measured every day and is not collected randomly, so there may be a random walk in the data.
% 
% Especially, weather, such as today's temperature depends on yesterday's temperature, which in turn affects tomorrow's weather.
% 
% Likewise, this might happen in the pattern of the food loss and waste.
% Possibly the volume of food waste may have some relationship to time.
% 
% Since most guests are regular customers, 
% we believe that their behaviour patterns are affected on a day-to-day basis.
% 
% If this is true, the regular regression analysis does not work on a time-series data, 
% and then it may result in the opposite results.
% 
% So, the next slide shows a danger of using regression analysis on time series data.

% \vfill\null
% \columnbreak
% \noindent\rule{0.2\textwidth}{1pt}(15)
% 
% (If we have time)
% 
% Let's create two fake data sets.
% 
% One data has just simply randomly generated data, 
% 
% and the other is a random-walk data which is dependent on one before.
% 
% Of course, both data, x and y, in both data sets do not have any relationship between each other.
% 
% And, if we take a linear regression model on the first data set, 
% of course, it does not have any relationship between x and y.
% 
% However, the second data set has an association 
% even if they do not formulate any relationship.
% 
% Time-series data, 
% i.e., when the currently collected data is dependent on one previous time, 
% in this case it is difficult to estimate a true value with the usual regression model.
% 
% \noindent\rule{0.2\textwidth}{1pt}(16)
% 
% Therefore, we will extend the fixed coefficient to a time-vary coefficient so that we can examine the two sequence data sets.
% 
% Below is a summary of the equation of the model and the results of the regression analysis using the data used earlier.
% 
% For the regression model, 
% the coefficients depend on one previous value.
% 
% In the graph on the right, 
% the top two graphs are almost identical and the last graph is the change in coefficients. 
% 
% This estimated coefficient is almost near zero, indicating that it does not have any relationship.
% 
% In other words, in this study, 
% we can avoid concluding that there is a relationship between food waste and weather or calendar effects,
% where there is no relationship at all.

% \vfill\null
% \columnbreak
\subsection*{\small Effects \noindent\rule{0.1\textwidth}{1pt}\small(15)}

After the regression analysis, 
this study will move on to the effects of food loss and waste.

The impacts will be examined in the following three categories.

\vspace{1em}

First, economic loss.

This mainly includes loss of resources, such as labour, material time and energy losses.

\vspace{1em}

Second, environmental effects.

I will research water pollution, soil erosion and greenhouse gas emissions.

\vspace{1em}

And last, social impacts.

This includes food insecurity and social inequality.

\vspace{1em}

By evaluating the estimated economic, environmental, and social impacts of Food loss and waste, 

it can be possible to lower those impacts through its reduction.

A greener and healthier society could be created through better supply chain management, 
reduced consumer food waste, and increase food recovery.

\subsection*{\small Goals \noindent\rule{0.1\textwidth}{1pt}\small(16)}

Therefore, the research will be focused on these four areas.

1. 
Estimation of food waste over the study period.

2. 
The relationship between food waste generation and business operations.

3. 
The relationship between food waste and weather and calendar.

4. 
And estimation of social or environmental impacts.

% \vfill\null
% \columnbreak
\subsection*{\small Method: Location \noindent\rule{0.1\textwidth}{1pt}\small(17)}

How I collect the food loss and waste?

The study site, which is a restaurant for data collection, 
is a Japanese restaurant in the suburban area of Prince George.

This site is a family-owned restaurant and open for lunch and dinner, each for three hours.

It is closed on Mondays and open the rest of the week, Tuesday to Sunday.

The items in the restaurant offer sushi and ramen (noodles and soup).

The collection period is 6 months: from September to March.

% As for the ramen, it typically serves about 900 grams per meal.

\vspace{1em}

In this study, 
permission and cooperation to measure the food loss and waste 
was obtained from restaurant owners.

However, I have not got permission to open the name of the restaurant, 
so I still withhold the name disclosure.

\subsection*{\small Apparatus \noindent\rule{0.1\textwidth}{1pt}\small(18)}

So, how do I collect food loss and waste?

Based on our food loss and waste definition,
this research uses two buckets and one strainer, a weight scale, 
and a pen and notebook for the record.

One bucket gets food loss, so, cooking scraps or peels from the in-house.

The second bucket with a strainer holds food leftover collected from the customers' side.

This strainer separates the solids from the liquid form of leftovers.

At the end of the business, each bucket is weighed using the scale.

After weighing them, I will record it in the notebook.

% \noindent\rule{0.2\textwidth}{1pt}(21)
% % Sample size
% 
% Well, how many samples, food loss and waste, should be collected?
% 
% The power analysis of the regression analysis shows that it needs 114 samples.
% 
% This is with a $90\%$ confidence interval, allowing for a $10\%$ margin of error and 10 independent variables.
% 
% The R code is written at the end.
% 
% Also, as a rule of thumb, 
% for a regression analysis with 10 explanatory variables,
% we have to have 150 due to the one-in-ten rule.
% 
% And over 130 samples must be collected, if following Green's rule, which is another rule.
% 
% So, I will collect more than 130 data in this research.

\subsection*{\small Variables \noindent\rule{0.1\textwidth}{1pt}\small(19)}

The following numbers are collected every single day.

From 1 to 3 are the dependent variables in the regression analysis 
and the rest are independent variables.

The number of customers is counted only those who came to eat at the restaurant.

Take-out is not counted.

Children are counted as one person.

Business variables show a change in management in the restaurant, a change in workers, or when a competing restaurant opens in the neighbourhood.

\subsection*{\small Regression model \noindent\rule{0.1\textwidth}{1pt}\small(20)}

The basic regression model is as follows.

The criteria for determining whether there is an association are
the confidence interval of the regression coefficients.

% \vfill\null
% \columnbreak
\subsection*{\small Expected Result \noindent\rule{0.1\textwidth}{1pt}\small(21)}


After collecting samples, 
this research will estimate and analyze food waste generated by one Japanese restaurant.

Therefore, this study will derive the following analytical results.

Estimation of food waste within the study period and
comparison with various factors and an estimation of their environmental and social impacts, 
and implications for food waste reduction.

\subsection*{\small Progress \noindent\rule{0.1\textwidth}{1pt}(22)}

This is current progress.

Food loss and waste collection began in September and is now in its sixth month.

More than 150 sample have been collected.

Also, I have created basic histograms and plots.

% \vfill\null
% \columnbreak
\subsection*{\small Plan \noindent\rule{0.2\textwidth}{1pt}\small(23)}


I will plan to carry out the following schedule in the future.

I will finish food loss and collection by the end of March.

After that, I will summarize my research findings and write up my master's thesis.

\noindent\rule{0.2\textwidth}{1pt}

That's all for my presentation.

Thank you for listening.

Any questions?

\end{spacing}
\end{multicols}
\end{document}